\begin{problem}{}{}{}{0.2 секунды}{64 мегабайта}
% Полустепени вершин по списку рёбер

Ориентированный граф задан списком рёбер. Найдите полустепени захода и полустепени исхода всех вершин графа.
Т.е. для каждой вершины надо посчитать сколько в неё входит рёбер и сколько из неё выходит рёбер.

\InputFile
В первой строке входных данных заданы числа $N$ ($1 \le N \le 100$) $-$ количество вершин в графе и 
$M$ ($1 \le M \le N\cdot (N-1)/2$) $-$ количество рёбер.
Далее следует $M$ строк по $2$ числа в каждой $-$ список рёбер графа. 

\OutputFile
Ваша программа должна вывести $N$ пар чисел (каждая пара в отдельной строке). Для каждой вершины сначала выведите 
полустепень захода и затем полустепень исхода.

\Example

\begin{example}
\exmp{4 4 
1 2 
1 3 
2 3 
3 4 
}{0 2 
1 1 
2 1 
1 0
}%
\end{example}

\end{problem}

