\begin{problem}{Наивный алгоритм - 2 (все вхождения)}{}{}{0.2 секунды}{64 мегабайта}

Даны строки $P$ и $T$. Найдите все вхождения строки $P$ в текст $T$, используя {\bf модификацию наивного алгоритма поиска}, 
при которой подстроки $T$ перебираются слева направо, а посимвольное сравнение строк выполняется справа налево.

\InputFile
В первой строке входных данных записан образец $P$, 
во второй строке записан текст $T$ ($1 \le |P| \le 100$, $1 \le |T| \le 100$).


\OutputFile
В первую строку выведите через пробел все позиции в $T$, с которых начинаются вхождения образца $P$. 
Если вхождений не было, то в первую строку выведите 0.
Во вторую строку выведите общее количество сравнений символов, которое потребовалось для решения задачи. 

\Examples

\begin{example}
\exmp{aba
aababac
}{2 4
9}%
\end{example}

\end{problem}
