\begin{problem}{}{}{}{0.2 секунды}{64 мегабайта}
% Истоки и стоки

Вершина ориентированного графа называется истоком, если в нее не входит ни одно ребро и стоком, если из нее не 
выходит ни одного ребра. 

Ориентированный граф задан матрицей смежности. Найдите все вершины графа, которые являются истоками, и все его вершины, 
которые являются стоками.

\InputFile
В первой строке входных данных задано число $N$ ($1 \le N \le 100$) $-$ количество вершин в графе.
Далее следует $N$ строк по $N$ чисел $-$ матрица смежности графа $G$. В матрице смежности элемент $G_{i,j}=1$, если
существует ребро, соединяющее вершины $i$ и $j$, $G_{i,j}=0$ $-$ в противном случае.

\OutputFile
В первой строке выведите число $k$ $-$ количество истоков в графе и затем $k$ чисел $-$ номера вершин, 
которые являются истоками, в возрастающем порядке. Во второй строке выведите информацию о стоках в таком же формате.

\Example
\begin{example}
\exmp{4 
1 0 0 1 
0 0 0 0 
1 1 0 1 
0 0 0 0
}{1 3 
2 2 4
}%
\end{example}

\end{problem}

