\begin{problem}{Наївний алгоритм - 2 (перше входження)}{}{}{0.2 секунди}{64 мегабайти}

Дані рядки $P$ і $T$. Знайдіть перше входження рядка $P$ в текст $T$, використовуючи {\bf модифікацію наївного алгоритму пошуку}, 
при якій підрядки $T$ перебираються зліва направо, а посимвольне порівняння рядків виконується справа наліво.

При кожному порівнянні символів (незалежно від успішності порівняння) потрібно виводити порівнюваний символ рядка $P$. 
Після завершення алгоритму потрібно вивести позицію в $T$, з якої починається перше входження $P$, або $0$, якщо входження відсутнє. 

\InputFile
У першому рядку вхідних даних записаний зразок $P$, 
у другому рядку записаний текст $T$ ($1 \le |P| \le 100$, $1 \le |T| \le 100$).


\OutputFile
У перший рядок виведіть порівнювані символи зразка $P$.
У другий рядок виведіть позицію першого входження зразка $P$ в текст $T$ або $0$, якщо входження відсутнє.


\Examples

\begin{example}
\exmp{acbab
baaaaaacbabcaacbaaba
}{bbbbbabbabca
7}%
\end{example}

\end{problem}

