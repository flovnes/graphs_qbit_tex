\begin{problem}{}{}{}{0.2 секунды}{64 мегабайта}
% Компоненты связности по матрице смежности

Дан неориентированный граф $G$ с $n$ вершинами и $m$ рёбрами. Требуется найти в нём все компоненты связности, 
т.е. разбить вершины графа на несколько групп так, что внутри одной группы можно дойти от одной вершины до любой другой, 
а между разными группами $-$ пути не существует.

  \begin{center}
    \includegraphics[width=0.40\textwidth,natwidth=232,natheight=217]{pic.png}
  \end{center}

Граф на иллюстрации содержит три компоненты связности, закрашенные разными цветами. 
Можно заметить, что даже одна вершина, изолированная от остального графа, составляет компоненту связности.

Общее понятие связности распространяется только на неориентированные графы.

\InputFile
В первой строке входных данных содержится одно натуральное число $N$ ($N \leqslant 100$) $-$ количество вершин в графе. 
Далее в $N$ строках по $N$ чисел $-$ матрица смежности графа: в $i$-й строке на $j$-м месте стоит \texttt{1}, если вершины $i$ и $j$ 
соединены ребром, и \texttt{0}, если ребра между ними нет. 
На главной диагонали матрицы стоят нули. Матрица симметрична относительно главной диагонали.

\OutputFile
Выведите количество компонент связности заданного графа.


\Example
\begin{example}
\exmp{6
0 1 1 0 0 0
1 0 1 0 0 0
1 1 0 0 0 0
0 0 0 0 1 0
0 0 0 1 0 0
0 0 0 0 0 0
}{3
}%
\end{example}

\end{problem}
