\begin{problem}{}{}{}{0.2 секунды}{64 мегабайта}
%Компоненти зв'язності по матриці суміжності

Дано не орієнтований граф $G$ з $n$ вершинами та $m$ ребрами. Потрібно знайти в ньому всі компоненти зв'язності, 
тобто розбити вершини графа на кілька груп так, що всередині однієї групи можна дійти від однієї вершини до будь-якої іншої,
а між різними групами $-$ шляха не існує.

  \begin{center}
    \includegraphics[width=0.40\textwidth,natwidth=232,natheight=217]{pic.png}
  \end{center}

Граф на ілюстрації містить три компоненти зв'язності, зафарбовані різними кольорами.
Можна помітити, що одна вершина, ізольована від решти графа, становить компоненту зв'язності.


Загальне поняття зв'язності поширюється лише на неорієнтовані графи.

\InputFile
У першому рядку вхідних даних міститься одне натуральне число $N$ ($N \leqslant 100$) $-$ количество вершин в графе.
Далі в $N$ рядках по $N$ чисел $-$ матриця суміжності графа: у $i$-у рядку на $j$-му місці стоїть \texttt{1}, якщо вершини $i$ і $j$ 
з'єднані ребром, і \texttt{0}, якщо ребра між ними немає. 
На головній діагоналі матриці стоять нулі. Матриця симетрична щодо головної діагоналі.

\OutputFile
Виведіть кількість компонентів зв'язності заданого графа.


\Example
\begin{example}
\exmp{6
0 1 1 0 0 0
1 0 1 0 0 0
1 1 0 0 0 0
0 0 0 0 1 0
0 0 0 1 0 0
0 0 0 0 0 0
}{3
}%
\end{example}

\end{problem}
