\begin{problem}{}{}{}{0.2 секунды}{64 мегабайта}
% Список рёбер ориентированного графа

Ориентированный граф задан матрицей смежности. Выведите его представление в виде списка ребер.

\InputFile
В первой строке входных данныъ задано число $N$ ($1 \le N \le 100$) $-$ количество вершин в графе.
Далее следует $N$ строк по $N$ чисел $-$ матрица смежности графа $G$. В матрице смежности элемент $G_{i,j}=1$, если
существует ребро, соединяющее вершины $i$ и $j$, $G_{i,j}=0$ $-$ в противном случае.

\OutputFile
Ваша программа должна вывести список рёбер заданного графа. Каждое ребро выводится так: сначала $-$
номер вершины, из которой ребро выходит, затем $-$ номер вершины, в которую ребро входит.

\Example

\begin{example}
\exmp{3
0 1 1 
0 0 1 
1 1 1 
}{1 2
1 3
2 3
3 1
3 2 
3 3
}%
\end{example}

\end{problem}

