\begin{problem}{}{}{}{0.2 секунди}{64 мегабайти}
% Список ребер орієнтованого графа

Орієнтований граф задано матрицею суміжності. Виведіть його подання у вигляді списку ребер.

\InputFile
У першому рядку вхідних даних задано число $N$ ($1 \le N \le 100$) $-$ кількість вершин у графі.
Далі слідує $N$ рядків по $N$ чисел $-$ матриця суміжності графа $G$. У матриці суміжності елемент $G_{i,j}=1$, якщо
існує ребро, що з'єднує вершини $i$ та $j$, $G_{i,j}=0$ $-$ в іншому випадку.

\OutputFile
Ваша програма має вивести список ребер заданого графа. Кожне ребро виводиться так: спочатку $-$
номер вершини, з якої ребро виходить, потім $-$ номер вершини, в яку ребро входить.

\Example

\begin{example}
\exmp{3
0 1 1 
0 0 1 
1 1 1 
}{1 2
1 3
2 3
3 1
3 2 
3 3
}%
\end{example}

\end{problem}

