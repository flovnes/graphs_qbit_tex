\begin{problem}{}{}{}{0.2 секунди}{64 мегабайти}
% Витоки і стоки

Вершина орієнтованого графа називається витоком, якщо в неї не входить жодне ребро і стоком, якщо з неї не 
виходить жодного ребра. 

Орієнтований граф задано матрицею суміжності. Знайдіть усі вершини графа, які є витоками, і всі його вершини, 
які є стоками.

\InputFile
У першому рядку вхідних даних задано число $N$ ($1 \le N \le 100$) $-$ кількість вершин у графі.
Далі слідує $N$ рядків по $N$ чисел $-$ матриця суміжності графа $G$. У матриці суміжності елемент $G_{i,j}=1$, якщо
існує ребро, що з'єднує вершини $i$ і $j$, $G_{i,j}=0$ $-$  у протилежному випадку.

\OutputFile
У першому рядку виведіть число $k$ $-$ кількість витоків у графі і потім $k$ чисел $-$ номери вершин, 
які є витоками, у зростаючому порядку. У другому рядку виведіть інформацію про стоки в такому ж форматі.

\Example
\begin{example}
\exmp{4 
1 0 0 1 
0 0 0 0 
1 1 0 1 
0 0 0 0
}{1 3 
2 2 4
}%
\end{example}

\end{problem}

