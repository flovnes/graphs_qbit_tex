\begin{problem}{}{}{}{0.2 секунды}{64 мегабайта}
% Количество рёбер неориентированного графа

{\bf Неориентированным графом} (англ. {\emph{undirected graph}}) называется пара $G=(V,E)$, 
где $V$ $-$ множество вершин, а $E \subset \{\{v,u\}: v,u \in V\}$ $-$ множество рёбер.
{\bf Ребром} в неориентированном графе называют неупорядоченную пару вершин $\{v,u\} \in E$.
Неориентированный граф называется простым, если в нём нет кратных рёбер и петель.

 \begin{center}
    \includegraphics[width=0.30\textwidth,natwidth=232,natheight=217]{pic.png}
  \end{center}

{\bf Матрица смежности} графа $G$ с конечным числом вершин $n$ (пронумерованных числами от $1$ до $n$) $-$ 
это квадратная матрица $A$ размера $n$, в которой значение элемента $G_{ij}$ равно числу рёбер из $i$-й вершины графа в $j$-ю вершину.
Матрица смежности простого неориентированного графа состоит из нулей и единиц, 
эта матрица симметрична относительно главной диагонали, на которой записаны нули ($G_{ij} = 0$).

Простой неориентированный граф задан матрицей смежности. Посчитайте количество рёбер в заданном графе.

\InputFile
В первой строке входных данных задано число $N$ ($1 \le N \le 100$) $-$ количество вершин в графе.
Далее следует $N$ строк по $N$ чисел $-$ матрица смежности графа $G$. В матрице смежности элемент $G_{i,j}=1$, если
существует ребро, соединяющее вершины $i$ и $j$, $G_{i,j}=0$ $-$ в противном случае.

\OutputFile
Ваша программа должна вывести одно число $-$ количество рёбер в графе.

\Example

\begin{example}
\exmp{4
0 1 0 1
1 0 1 1
0 1 0 0
1 1 0 0
}{4
}%
\end{example}

\end{problem}

