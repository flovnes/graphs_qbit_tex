\begin{problem}{}{}{}{0.2 секунди}{64 мегабайти}
% Кількість ребер неорієнтованого графа

{\bf Неорієнтованим графом} (англ. {\emph{undirected graph}}) називається пара $G=(V,E)$, 
де $V$ $-$ множина вершин, а $E \subset \{\{v,u\}: v,u \in V\}$ $-$ множина ребер.
{\bf Ребром} у неорієнтованому графі називають невпорядковану пару вершин $\{v,u\} \in E$.
Неорієнтований граф називається простим, якщо в ньому немає кратних ребер і петель.

 \begin{center}
    \includegraphics[width=0.30\textwidth,natwidth=232,natheight=217]{pic.png}
  \end{center}

{\bf Матриця суміжності} графа $G$ зі скінченним числом вершин $n$ (пронумерованих числами від $1$ до $n$) $-$ 
це квадратна матриця $A$ розміром $n$, в якій значення елемента $G_{ij}$ дорівнює числу ребер з $i$-ї вершини графа в $j$-ю вершину.
Матриця суміжності простого неорієнтованого графа складається з нулів і одиниць,
ця матриця симетрична щодо головної діагоналі, на якій записані нулі ($G_{ij} = 0$).

Простий неорієнтований граф задано матрицею суміжності. Порахуйте кількість ребер у заданому графі.

\InputFile
У першому рядку вхідних даних задано число $N$ ($1 \le N \le 100$) $-$ кількість вершин у графі.
Далі йде $N$ рядків по $N$ чисел $-$ матриця суміжності графа $G$. У матриці суміжності елемент $G_{i,j}=1$, якщо
існує ребро, що з'єднує вершини $i$ і $j$, $G_{i,j}=0$ $-$ в іншому випадку.

\OutputFile
Ваша програма має вивести одне число $-$ кількість ребер у графі.

\Example

\begin{example}
\exmp{4
0 1 0 1
1 0 1 1
0 1 0 0
1 1 0 0
}{4
}%
\end{example}

\end{problem}

