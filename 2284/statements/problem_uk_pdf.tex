\begin{problem}{Префиксы}{}{}{0.2 секунды}{64 мегабайта}

Префиксом строки $S$ называется любая ее подстрока вида $S[1..k]$, $k \le |S|$ 
($|S|$ обозначает длину строки $S$). 
Префикс может быть как пустой строкой, так и совпадать со всей строкой.
Если префикс строки $S$ является непустым и не совпадает со всей строкой, 
то он называется {\it собственным префиксом}. 

Напишите программу, которая выводит количество префиксов подстроки $S[i..j]$, а также 
все собственные префиксы в порядке возрастания их длин.

\InputFile
Во входных данных записана строка $S$ и два индекса $i$ и $j$. Длина строки $S$ не превосходит $100$, $1 \le i \le |S|$, $1 \le j \le |S|$.

\OutputFile
Выведите количество перфиксов подстроки $S[i..j]$
и все собственные префиксы в порядке возрастания их длин.

\Examples

\begin{example}
\exmp{abracadabra
2 5
}{5
b
br
bra}%
\exmp{abracadabra
1 3
}{4
a
ab
}%
\end{example}

\end{problem}
