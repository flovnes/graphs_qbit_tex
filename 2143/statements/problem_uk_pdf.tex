\begin{problem}{}{}{}{0.2 секунди}{64 мегабайти}
% Кількість ребер неорієнтованого графа

{\bf {Орієнтований граф}} (коротко орграф) $-$ (мульти) граф, ребрам якого присвоєно напрямок. 
Спрямовані ребра іменуються також дугами, а в деяких джерелах і просто ребрами. 

{\bf {Орієнтованим графом}} (англ. {\emph{directed graph}}) називається пара $G=(V,E)$, де $V$ $-$ множина вершин 
(англ. {\emph{vertices}}), а $E \subset V \times V$ $-$ множина ребер. 
Ребром (англ. {\emph{edge}}, дугою (англ. {\emph{arc}}), лінією (англ. {\emph{line}})) орієнтованого графа 
називають упорядковану пару вершин $(v,u) \subset E$.

 \begin{center}
    \includegraphics[width=0.50\textwidth,natwidth=232,natheight=217]{pic.png}
  \end{center}


Орієнтований граф задано матрицею суміжності. Порахуйте кількість ребер у заданому графі.

\InputFile
У першому рядку вхідних даних задано число $N$ ($1 \le N \le 100$) $-$ кількість вершин у графі.
Далі йде $N$ рядків по $N$ чисел $-$ матриця суміжності графа $G$. У матриці суміжності елемент $G_{i,j}=1$, якщо
існує ребро, що з'єднує вершини $i$ и $j$, $G_{i,j}=0$ $-$ в іншому випадку.

\OutputFile
Ваша програма має вивести одне число $-$ кількість ребер у графі.

\Example

\begin{example}
\exmp{4
0 1 0 1
1 0 0 1
0 1 0 0
1 0 1 1
}{8
}%
\end{example}

\end{problem}

