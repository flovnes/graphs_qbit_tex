\begin{problem}{}{}{}{0.2 секунды}{64 мегабайта}
% Количество рёбер ориентированного графа

{\bf {Ориентированный граф}} (кратко орграф) $-$ (мульти) граф, рёбрам которого присвоено направление. 
Направленные рёбра именуются также дугами, а в некоторых источниках и просто рёбрами. 

{\bf {Ориентированным графом}} (англ. {\emph{directed graph}}) называется пара $G=(V,E)$, где $V$ $-$ множество вершин 
(англ. {\emph{vertices}}), а $E \subset V \times V$ $-$ множество рёбер. 
Ребром (англ. {\emph{edge}}, дугой (англ. {\emph{arc}}), линией (англ. {\emph{line}})) ориентированного графа 
называют упорядоченную пару вершин $(v,u) \subset E$.

 \begin{center}
    \includegraphics[width=0.50\textwidth,natwidth=232,natheight=217]{pic.png}
  \end{center}


Oриентированный граф задан матрицей смежности. Посчитайте количество рёбер в заданном графе.

\InputFile
В первой строке входных данных задано число $N$ ($1 \le N \le 100$) $-$ количество вершин в графе.
Далее следует $N$ строк по $N$ чисел $-$ матрица смежности графа $G$. В матрице смежности элемент $G_{i,j}=1$, если
существует ребро, соединяющее вершины $i$ и $j$, $G_{i,j}=0$ $-$ в противном случае.

\OutputFile
Ваша программа должна вывести одно число $-$ количество рёбер в графе.

\Example

\begin{example}
\exmp{4
0 1 0 1
1 0 0 1
0 1 0 0
1 0 1 1
}{8
}%
\end{example}

\end{problem}

