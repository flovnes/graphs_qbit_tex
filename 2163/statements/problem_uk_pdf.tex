\begin{problem}{}{}{}{0.2 секунди}{64 мегабайти}

Неорієнтований зважений граф задано матрицею суміжності. Знайдіть у цьому графі всі ребра з максимальною і мінімальною вагою.

\InputFile
У першому стоці вхідних даних задано число $N$ ($1 \le N \le 100$) $-$ кількість вершин у графі.
Далі йде $N$ рядків по $N$ чисел $-$ матриця суміжності графа $G$.
У матриці суміжності елемент $G_{i,j} \ne 0$, якщо існує ребро, що з'єднує вершини $i$ і $j$, $G_{i,j}=0$ $-$ в іншому випадку.
Елемент $G_{i,j}$ у матриці суміжності означає вагу ребра, що з'єднує вершини з номерами $i$ і $j$.

\OutputFile
Виведіть спочатку всі ребра, що мають максимальну вагу, а потім всі ребра, що мають мінімальну вагу.
Кожну пару вершин виводьте в окремому рядку. У кожній парі спочатку виводиться вершина з меншим номером.
Нумерація вершин починається з одиниці.

Пари, що відповідають за ребра, які мають максимальну вагу, мають бути відсортовані в порядку зростання.
Пари, що відповідають за ребра, які мають мінімальну вагу, мають бути відсортовані в порядку спадання.
Пари вершин порівнюються так: 
$$ (i_1, j_1) > (i_2, j_2) \Leftrightarrow (i_1 > i_2) \ or \ ((i_1=i_2) \ and \ (j_1>j_2))$$

\Example

\begin{example}
\exmp{5
0 2 0 5 5 
2 0 3 0 0 
0 3 0 2 2 
5 0 2 0 4 
5 0 2 4 0 
}{
1 4
1 5
3 5
3 4
1 2
}
\end{example}

\end{problem}
