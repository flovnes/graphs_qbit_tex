\begin{problem}{}{}{}{0.2 секунди}{64 мегабайти}

Простий неорієнтований граф задано матрицею суміжності. Кожна вершина графа пофарбована в певний колір.
Порахуйте кількість ребер, які з'єднують різнокольорові вершини.

\InputFile
У першому стоці вхідних даних задано число $N$ ($1 \le N \le 100$) $-$ кількість вершин у графі.
Далі йде $N$ рядків по $N$ чисел $-$ матриця суміжності графа $G$.
У матриці суміжності елемент $G_{i,j}=1$, якщо існує ребро, що з'єднує вершини $i$ і $j$, $G_{i,j}=0$ $-$ в іншому випадку.
У наступному рядку записано $N$ чисел з діапазону від $1$ до $10$ $-$ кольору вершин.

\OutputFile
Виведіть одне число $-$ кількість ребер, які з'єднують різнокольорові вершини.

\Example

\begin{example}
\exmp{7
0 1 0 0 0 1 1
1 0 1 0 0 0 0
0 1 0 0 1 1 0
0 0 0 0 0 0 0
0 0 1 0 0 1 0
1 0 1 0 1 0 0
1 0 0 0 0 0 0
1 1 1 1 1 3 3
}{4
}
\end{example}

\end{problem}
