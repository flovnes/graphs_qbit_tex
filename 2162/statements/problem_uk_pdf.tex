\begin{problem}{}{}{}{0.2 секунды}{64 мегабайта}
% Разноцветные вершины

Простой неориентированный граф задан матрицей смежности. Каждая вершина графа покрашена в определённый цвет.
Посчитайте количество рёбер, которые соединяют разноцветные вершины.

\InputFile
В первой стоке входных данных задано число $N$ ($1 \le N \le 100$) $-$ количество вершин в графе.
Далее следует $N$ строк по $N$ чисел $-$ матрица смежности графа $G$. В матрице смежности элемент $G_{i,j}=1$, если
существует ребро, соединяющее вершины $i$ и $j$, $G_{i,j}=0$ $-$ в противном случае.
В следующей строке записаны $N$ чисел из диапазона от $1$ до $10$ $-$ цвета вершин.

\OutputFile
Выведите одно число $-$ количество рёбер, которые соединяют разноцветные вершины.

\Example

\begin{example}
\exmp{7
0 1 0 0 0 1 1
1 0 1 0 0 0 0
0 1 0 0 1 1 0
0 0 0 0 0 0 0
0 0 1 0 0 1 0
1 0 1 0 1 0 0
1 0 0 0 0 0 0
1 1 1 1 1 3 3
}{4
}%
\end{example}

\end{problem}

